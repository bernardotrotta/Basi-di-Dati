% Documento principale
\documentclass{article}
\usepackage[utf8]{inputenc}
\usepackage{flafter} 

\title{Algebra relazionale}
\author{}
\date{}

\begin{document}

\maketitle

\section{Algebra relazionale}

L'algebra relazionale è un linguaggio procedurale utilizzato nell'ambito delle basi di dati per costruire interrogazioni; si parla quindi di linguaggi di interrogazione.

\begin{itemize}
  \item È costituita da un insieme di \textit{operatori} definiti su \textit{relazioni}.
  \item Produce una nuova relazione a sua volta.
\end{itemize}

\section{Operatori}

\subsection{Operatori insiemistici}
\begin{enumerate}
  \item Unione
  \item Intersezione
  \item Differenza
\end{enumerate}

\subsection*{Operatori specifici}
\begin{enumerate}
  \item Ridenominazione
  \item Selezione
  \item Proiezione
\end{enumerate}

\subsection*{Join}
\begin{enumerate}
  \item Join Naturale
  \item Prodotto cartesiano
  \item Theta Join
\end{enumerate}

\subsection*{Operatori insiemistici}
Le relazioni sono insiemi di tuple omogenee; pertanto, è necessario ereditare le operazioni fondamentali della teoria degli insiemi.
Importante notare che le relazioni devono essere definite sugli stessi attributi.

\begin{table}[htbp]
\centering
\begin{tabular}{|c|c|c|}
\hline
Matricola & Cognome & Età \\ \hline
7274      & Rossi   & 37  \\ \hline
7432      & Neri    & 39  \\ \hline
9824      & Verdi   & 28  \\ \hline
\end{tabular}
\end{table}

\begin{table}[htbp]
\centering
\begin{tabular}{|c|c|c|}
\hline
Matricola & Cognome & Età \\ \hline
9297      & Neri    & 56  \\ \hline
7432      & Neri    & 39  \\ \hline
9824      & Verdi   & 38  \\ \hline
\end{tabular}
\end{table}

\subsubsection*{Unione}
L'unione di due relazioni $r_1$ e $r_2$ definite sullo stesso insieme $X$ di attributi è una relazione ancora su $X$ contenente le tuple che appartengono a $r_1$, oppure ad $r_2$, oppure ad entrambe. È indicata con $r_1 \cup r_2$.

\begin{table}[htbp]
\centering
\begin{tabular}{|c|c|c|}
\hline
Matricola & Cognome & Età \\ \hline
7274      & Rossi   & 37  \\ \hline
7432      & Neri    & 39  \\ \hline
9824      & Verdi   & 28  \\ \hline
9297      & Neri    & 56  \\ \hline
\end{tabular}
\end{table}

\subsubsection*{Intersezione: $r_1 \cap r_2$}
L'intersezione di $r_1(X)$ e $r_2(X)$ è una relazione su $X$ contenente le tuple che appartengono sia a $r_1$ sia a $r_2$. $r_1 \cap r_2$.

\begin{table}[htbp]
\centering
\begin{tabular}{|c|c|c|}
\hline
Matricola & Cognome & Età \\ \hline
7432      & Neri    & 39  \\ \hline
9824      & Verdi   & 28  \\ \hline
\end{tabular}
\end{table}

\subsubsection*{Differenza}
La differenza di $r_1(X)$ e $r_2(X)$ è una relazione su $X$ contenente le tuple che appartengono ad $r_1$ e non appartengono ad $r_2$. È indicata con $r_1 - r_2$.

\begin{table}[htbp]
\centering
\begin{tabular}{|c|c|c|}
\hline
Matricola & Cognome & Età \\ \hline
7274      & Rossi   & 37  \\ \hline
\end{tabular}
\end{table}

\subsection*{Operatori specifici}

\subsubsection*{Ridenominazione}
Cambia i nomi degli attributi a seconda delle necessità, al fine di superare le limitazioni imposte agli operatori insiemistici. Da notare che la ridenominazione non cambia il contenuto della relazione e agisce solo sullo schema.

\begin{table}[htbp]
\centering
\begin{tabular}{|c|c|}
\hline
Padre & Figlio \\ \hline
Madre & Figlio \\ \hline
\end{tabular}
\end{table}

\end{document}
