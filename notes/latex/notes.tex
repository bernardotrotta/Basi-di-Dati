\documentclass{article}

\begin{document}

Atzeni, Ceri. Fraternali, Paraboschi, Torlone Basi di Dati, 6a ed., McGraw-Hill 2023

\section{Strumenti di studio}
\begin{itemize}
  \item Novella matheducation
  \item Guardare gli appunti
\end{itemize}

\section{Dato}
L'informazione è tutto ciò che riceviamo e ci aiuta a migliorare la conoscenza sul mondo che ci circonda. Un dato è la formalizzazione della informazione che ci raggiunge e può essere espresso sottoforma di linguaggio. A livello informatico un dato è una informazione utilizzabile in un qualche modo all'interno del calcolatore. Una base di dati è una collezione di dati strutturata secondo un modello scelto (relazionale, logico, ecc...).

\section{Database Managment System (DBMS)}
Sistema software che permette di gestire le basi di dati. È formalmente visto come una estensione delle funzionalità di un file system. Il concetto di \textbf{indipendenza dei dati} sviluppando un'applicazione da zero i dati restano confinati sul DBMS. Un altro vantaggio sta nel poter aggiornare l'applicazione senza dover aggiornare tutti i dati. Le basi di dati gestite dai DBMS sono collezioni di dati.

\subsection{Pro}
\begin{itemize}
  \item Le applicazioni possono accedere al database in maniera indipendente.
  \item Le informazioni che abbiamo sul DBMS non sparisco più se l'applicazione non è in esecuzione.
\end{itemize}

\subsection{Contro}
\begin{itemize}
  \item Cambiando la base di dati potrei avere la necessità di ristrutturare tutte le applicazioni che fanno riferimento a quel database. Il problema è però facilmente risolvibile utilizzando le "liste" (Questa parte è da rivedere).
\end{itemize}

\section{Caratteristiche di un DBMS}
\begin{itemize}
  \item Affidabilità: non devo perdere i dati.
  \item Privatezza dei dati: Garantire dei privilegi a certe categorie di utenti.
  \item Efficienza: Capacità di eseguire in modo rapido le operazioni richieste.
  \item Efficacia: Capacità di eseguire in modo corretto le operazioni richieste.
\end{itemize}

\section{Architettura Client-Server}
Utente, Applicazione, Richiesta al Server, Richiesta all'interprete SQL, Risposta HTML

\section{Modelli dei dati}
Di seguito le fasi della progettazione di una base dei dati:
\begin{enumerate}
  \item Raccolta delle specifiche
  \item Progettazione concettuale
  \item Progettazione logica
  \item Progettazione fisica
\end{enumerate}

\subsection{Progettazione concettuale}
Modello e diagramma entità-relazione, bisogna rappresentare ogni concetto e le informazioni che servono per caratterizzarlo in modo completo. Esame collega il concetto di studente al concetto di insegnamento.

\textbf{Esempio:} Esame = Relazione, Studente = Concetto, Insegnamento = Concetto

\textbf{Nota:} Il diagramma entità relazione prevede il concetto di \textit{ereditarietà}.

Esiste il concetto di generalizzazione in cui è possibile gerarchizzare e specializzare i concetti.

Riodinamento delle specifiche in funzione del modello dei dati che utilizzo.

\subsection{Modelli concettuali}
Diagrammi di flusso. Descrivono la realtà mediante concetti astratti ma sono soggetti a precise regole. Descrivono i concetti del mondo reale. Il modello concettuale che useremo è il modello entità/relazione.
\begin{itemize}
  \item Attributi multi-valore: a livello concettuale un concetto è a sua volta suddiviso in più sotto concetti: prendendo il caso del concetto di indirizzo, si ha la via, codice civico, interno, codice postale, ecc...
\end{itemize}

\subsection{Modello relazionale}
In informatica il modello relazionale è una teoria matematica che offre gli strumenti concettuali per modellare una base di dati in termini di valori atomici e relazioni tra di essi.

Codice utilizzato nel calcolatore:
\begin{itemize}
  \item Non valgono gli attributi multi-valore.
\end{itemize}

\section{Problema fondamentale}
Descrivere una realtà isolata rispetto al resto del mondo nel modo più completo possibile, identificando tutti i concetti che lo compongono, attributi compresi. Importante eliminare il superfluo, se tutti gli elementi fanno riferimento ad una sola realtà, non è necessario identificare quest'ultima come concetto.

Nel momento in cui identifico più concetti, è necessario che ci sia almeno una relazione tra loro.

Un concetto che collega altri concetti e che quindi non può esistere se non esistono i concetti che collega si chiama relazione o associazione.

- Una relazione non è banalmente definibile soltanto attraverso i riferimenti agli altri oggetti, ma anche essa è caratterizzata da specifici attributi.

\textbf{Esempio:} Se la vendità è la \textit{relazione}, la sua descrizione dovrà contenere l'informazione relativa al cliente cui viene venduto un prodotto e al prodotto che ne è soggetto.

Nel caso del modello relazionale, ogni istanza del concetto (ogni riga della tabella) dovrà essere diversa da tutte le altre.

Nel modello concettuale entità relazione, i concetti vengono rappresentati con rettangoli, la relazione con dei rombi, la coppia numerica indica numero minimo e numero massimo in cui una istanza di una certa entità è coinvolta in una relazione.


Quei valori si chiamano cardinalità minima o numero di occorrenza minima.

\section{Progettazione logica e Modelli logici}
A differenza dei modelli concettuali, questi descrivono la struttura con cui i dati sono organizzati all'interno del calcolatore. Fanno riferimento a concetti che sono ancora astratti, compatibili però con le strutture dati di un calcolatore. Di seguito i modelli logici più utilizzati:
\begin{enumerate}
  \item \textbf{Relazionale:} basato su un modello tabellare dei dati.
  \item \textbf{Gerarchico:} basato su strutture ad albero, utilizzato nei primi DBMS (anni 60).
  \item \textbf{Reticolare:} basato sui grafi, estende il modello gerarchico.
  \item \textbf{A oggetti:} basato sui paradigmi della \textit{programmazione ad oggetti}, estende il modello relazionale.
  \item \textbf{XML (semistrutturato):} Deriva dal modello gerarchico ma è più flessibile. Viene considerato modello non relazionale
\end{enumerate}

\section{title}

\end{document}
\section{Strumenti di studio}
\begin{itemize}
    \item Novella matheducation
    \item Guardare gli appunti
\end{itemize}

\section{Dato}
L'informazione è tutto ciò che riceviamo e ci aiuta a migliorare la conoscenza sul mondo che ci circonda. Un dato è la formalizzazione della informazione che ci raggiunge e può essere espresso sottoforma di linguaggio. A livello informatico un dato è una informazione utilizzabile in un qualche modo all'interno del calcolatore. Una base di dati è una collezione di dati strutturata secondo un modello scelto (relazionale, logico, ecc...).

\section{Database Managment System (DBMS)}
Sistema software che permette di gestire le basi di dati. È formalmente visto come una estensione delle funzionalità di un file system. Il concetto di \textbf{indipendenza dei dati} sviluppando un'applicazione da zero i dati restano confinati sul DBMS. Un altro vantaggio sta nel poter aggiornare l'applicazione senza dover aggiornare tutti i dati. Le basi di dati gestite dai DBMS sono collezioni di dati.

\subsection{Pro}
\begin{itemize}
    \item Le applicazioni possono accedere al database in maniera indipendente.
    \item Le informazioni che abbiamo sul DBMS non sparisco più se l'applicazione non è in esecuzione.
\end{itemize}

\subsection{Contro}
\begin{itemize}
    \item Cambiando la base di dati potrei avere la necessità di ristrutturare tutte le applicazioni che fanno riferimento a quel database. Il problema è però facilmente risolvibile utilizzando le "liste" (Questa parte è da rivedere).
\end{itemize}

\section{Caratteristiche di un DBMS}
\begin{itemize}
    \item Affidabilità: non devo perdere i dati.
    \item Privatezza dei dati: Garantire dei privilegi a certe categorie di utenti.
    \item Efficienza: Capacità di eseguire in modo rapido le operazioni richieste.
    \item Efficacia: Capacità di eseguire in modo corretto le operazioni richieste.
\end{itemize}

\section{Architettura Client-Server}
Utente, Applicazione, Richiesta al Server, Richiesta all'interprete SQL, Risposta HTML

\section{Modelli dei dati}
Di seguito le fasi della progettazione di una base dei dati:
\begin{enumerate}
    \item Raccolta delle specifiche
    \item Progettazione concettuale
    \item Progettazione logica
    \item Progettazione fisica
\end{enumerate}

\subsection{Progettazione concettuale}
Modello e diagramma entità-relazione, bisogna rappresentare ogni concetto e le informazioni che servono per caratterizzarlo in modo completo. Esame collega il concetto di studente al concetto di insegnamento.

\textbf{Esempio:} Esame = Relazione, Studente = Concetto, Insegnamento = Concetto

\textbf{Nota:} Il diagramma entità relazione prevede il concetto di \textit{ereditarietà}.

Esiste il concetto di generalizzazione in cui è possibile gerarchizzare e specializzare i concetti.

Riodinamento delle specifiche in funzione del modello dei dati che utilizzo.

\subsection{Modelli concettuali}
Diagrammi di flusso. Descrivono la realtà mediante concetti astratti ma sono soggetti a precise regole. Descrivono i concetti del mondo reale. Il modello concettuale che useremo è il modello entità/relazione.
\begin{itemize}
    \item Attributi multi-valore: a livello concettuale un concetto è a sua volta suddiviso in più sotto concetti: prendendo il caso del concetto di indirizzo, si ha la via, codice civico, interno, codice postale, ecc...
\end{itemize}

\subsection{Modello relazionale}
In informatica il modello relazionale è una teoria matematica che offre gli strumenti concettuali per modellare una base di dati in termini di valori atomici e relazioni tra di essi.

Codice utilizzato nel calcolatore:
\begin{itemize}
    \item Non valgono gli attributi multi-valore.
\end{itemize}

\section{Problema fondamentale}
Descrivere una realtà isolata rispetto al resto del mondo nel modo più completo possibile, identificando tutti i concetti che lo compongono, attributi compresi. Importante eliminare il superfluo, se tutti gli elementi fanno riferimento ad una sola realtà, non è necessario identificare quest'ultima come concetto.

Nel momento in cui identifico più concetti, è necessario che ci sia almeno una relazione tra loro.

Un concetto che collega altri concetti e che quindi non può esistere se non esistono i concetti che collega si chiama relazione o associazione.

- Una relazione non è banalmente definibile soltanto attraverso i riferimenti agli altri oggetti, ma anche essa è caratterizzata da specifici attributi.

\textbf{Esempio:} Se la vendità è la \textit{relazione}, la sua descrizione dovrà contenere l'informazione relativa al cliente cui viene venduto un prodotto e al prodotto che ne è soggetto.

Nel caso del modello relazionale, ogni istanza del concetto (ogni riga della tabella) dovrà essere diversa da tutte le altre.

Nel modello concettuale entità relazione, i concetti vengono rappresentati con rettangoli, la relazione con dei rombi, la coppia numerica indica numero minimo e numero massimo in cui una istanza di una certa entità è coinvolta in una relazione.


Quei valori si chiamano cardinalità minima o numero di occorrenza minima.

\section{Progettazione logica e Modelli logici}
A differenza dei modelli concettuali, questi descrivono la struttura con cui i dati sono organizzati all'interno del calcolatore. Fanno riferimento a concetti che sono ancora astratti, compatibili però con le strutture dati di un calcolatore. Di seguito i modelli logici più utilizzati:
\begin{enumerate}
    \item \textbf{Relazionale:} basato su un modello tabellare dei dati.
    \item \textbf{Gerarchico:} basato su strutture ad albero, utilizzato nei primi DBMS (anni 60).
    \item \textbf{Reticolare:} basato sui grafi, estende il modello gerarchico.
    \item \textbf{A oggetti:} basato sui paradigmi della \textit{programmazione ad oggetti}, estende il modello relazionale.
    \item \textbf{XML (semistrutturato):} Deriva dal modello gerarchico ma è più flessibile. Viene considerato modello non relazionale
\end{enumerate}

\section{title}

\end{document}
